
% ------------------------------------------------------------------------
% ------------------------------------------------------------------------
% abnTeX2: Modelo de Artigo Acadêmico em conformidade com
% ABNT NBR 6022:2003: Informação e documentação - Artigo em publicação 
% periódica científica impressa - Apresentação
% ------------------------------------------------------------------------
% ------------------------------------------------------------------------

\documentclass[
	% -- opções da classe memoir --
	article,			% indica que é um artigo acadêmico
	12pt,				% tamanho da fonte
	oneside,			% para impressão apenas no verso. Oposto a twoside
	a4paper,			% tamanho do papel. 
	% -- opções da classe abntex2 --
	%chapter=TITLE,		% títulos de capítulos convertidos em letras maiúsculas
	%section=TITLE,		% títulos de seções convertidos em letras maiúsculas
	%subsection=TITLE,	% títulos de subseções convertidos em letras maiúsculas
	%subsubsection=TITLE % títulos de subsubseções convertidos em letras maiúsculas
	% -- opções do pacote babel --
	english,			% idioma adicional para hifenização
	brazil,				% o último idioma é o principal do documento
	sumario=tradicional
	]{abntex2}


% ---
% PACOTES
% ---

% ---
% Pacotes fundamentais 
% ---
\usepackage{lmodern}			% Usa a fonte Latin Modern
\usepackage[T1]{fontenc}		% Selecao de codigos de fonte.
\usepackage[utf8]{inputenc}		% Codificacao do documento (conversão automática dos acentos)
\usepackage{indentfirst}		% Indenta o primeiro parágrafo de cada seção.
\usepackage{nomencl} 			% Lista de simbolos
\usepackage{color}				% Controle das cores
\usepackage{graphicx}			% Inclusão de gráficos
\usepackage{microtype} 			% para melhorias de justificação
% ---
		
% ---
% Pacotes adicionais, usados apenas no âmbito do Modelo Canônico do abnteX2
% ---
\usepackage{lipsum}				% para geração de dummy text
% ---
		
% ---
% Pacotes de citações
% ---
\usepackage[brazilian,hyperpageref]{backref}	 % Paginas com as citações na bibl
\usepackage[alf]{abntex2cite}	% Citações padrão ABNT
% ---

% ---
% Configurações do pacote backref
% Usado sem a opção hyperpageref de backref
\renewcommand{\backrefpagesname}{Citado na(s) página(s):~}
% Texto padrão antes do número das páginas
\renewcommand{\backref}{}
% Define os textos da citação
\renewcommand*{\backrefalt}[4]{
	\ifcase #1 %
		Nenhuma citação no texto.%
	\or
		Citado na página #2.%
	\else
		Citado #1 vezes nas páginas #2.%
	\fi}%
% ---
%
%
%
% ---
% Informações de dados para CAPA e FOLHA DE ROSTO
% ---
\titulo{Revisão de Genética do Comportmanto Animal}
\autor{Josiel Patricio Pereira de Oliveira}
\local{ UFRN, Brasil}
%\data{2018}
% ---

% ---
% Configurações de aparência do PDF final

% alterando o aspecto da cor azul
\definecolor{blue}{RGB}{41,5,195}

% informações do PDF
\makeatletter
\hypersetup{
     	%pagebackref=true,
		pdftitle={\@title}, 
		pdfauthor={\@author},
    	pdfsubject={Modelo de artigo científico com abnTeX2},
	    pdfcreator={LaTeX with abnTeX2},
		pdfkeywords={abnt}{latex}{abntex}{abntex2}{atigo científico}, 
		colorlinks=true,       		% false: boxed links; true: colored links
    	linkcolor=blue,          	% color of internal links
    	citecolor=blue,        		% color of links to bibliography
    	filecolor=magenta,      		% color of file links
		urlcolor=blue,
		bookmarksdepth=4
}
\makeatother
% --- 

% ---
% compila o indice
% ---
\makeindex
% ---

% ---
% Altera as margens padrões
% ---
\setlrmarginsandblock{3cm}{3cm}{*}
\setulmarginsandblock{3cm}{3cm}{*}
\checkandfixthelayout
% ---

% --- 
% Espaçamentos entre linhas e parágrafos 
% --- 

% O tamanho do parágrafo é dado por:
\setlength{\parindent}{1.3cm}

% Controle do espaçamento entre um parágrafo e outro:
\setlength{\parskip}{0.2cm}  % tente também \onelineskip

% Espaçamento simples
\SingleSpacing

% ----
% Início do documento
% ----
\begin{document}

% Retira espaço extra obsoleto entre as frases.
\frenchspacing 

% ----------------------------------------------------------
% ELEMENTOS PRÉ-TEXTUAIS
% ----------------------------------------------------------

%---
%
% Se desejar escrever o artigo em duas colunas, descomente a linha abaixo
% e a linha com o texto ``FIM DE ARTIGO EM DUAS COLUNAS''.
% \twocolumn[    		% INICIO DE ARTIGO EM DUAS COLUNAS
%
%---
% página de titulo
\maketitle

% resumo em português
\begin{resumoumacoluna}

Este trabalho tem como objetivo ser uma revisão sobre os conteúdos ministrados na disciplina ``Genética do Comportamento Animal''. Disciplina desenvolvida ao longo do tempo motivada pela vontade do homem de entender o comportamento e os mecanismos biológicos com ele envolvidos. Entender a influência da genética e do meio e o quanto cada um interfere na expressão do comportamento.
 %Conforme a ABNT NBR 6022:2003, o resumo é elemento obrigatório, constituído de  uma sequência de frases concisas e objetivas e não de uma simples enumeração  de tópicos, não ultrapassando 250 palavras, seguido, logo abaixo, das palavras  representativas do conteúdo do trabalho, isto é, palavras-chave e/ou  descritores, conforme a NBR 6028. (\ldots) As palavras-chave devem figurar logo  abaixo do resumo, antecedidas da expressão Palavras-chave:, separadas entre si por ponto e finalizadas também por ponto.
 
 \vspace{\onelineskip}
 
 \noindent
 \textbf{Palavras-chaves}: genética do comportamento. comportamento animal.
\end{resumoumacoluna}

% ]  				% FIM DE ARTIGO EM DUAS COLUNAS
% ---

% ----------------------------------------------------------
% ELEMENTOS TEXTUAIS
% ----------------------------------------------------------
\textual

% ----------------------------------------------------------
% Introdução
% ----------------------------------------------------------
\section{Introdução}
\label{intro}
\addcontentsline{toc}{section}{Introdução}
Algumas reflexões importantes que visam ser respondidas pelo estudo da genética do comportamento animal são levantadas por Kandel já no prefácio de sua obra ``\textit{Principles of neural science}'' como, por exemplo: O que os genes contribuem para o comportamento e como a expressão gênica nas células nervosas é regulada pelos processos de desenvolvimento e aprendizado? Como a experiência altera a forma como o cérebro processa eventos subsequentes e até que ponto esse processamento é inconsciente?

Durante a segunda metade do século XX, o foco central da biologia estava no gene. Agora, na primeira metade do século XXI, o foco central da biologia mudou para a ciência neural e especificamente para a biologia da mente.
\cite{kandel2000principles} Ao longo dessa revisão serão abordados temas que visam explicar melhor questões de interações genética-ambiente-comportamento.
% ----------------------------------------------------------
% Seção de explicações
% ----------------------------------------------------------
\section{Assuntos apresentados em aulas}
\label{aulas}
A genética do comportamental visa compreender os mecanismos genéticos que permitem ao sistema nervoso direcionar interações apropriadas entre os organismos e seus ambientes sociais e físicos, demonstrando seu caráter multidisciplinar.
As primeiras explorações científicas do comportamento animal definiram os campos da  psicologia do comportamento e da etologia clássica \cite{anholt2009principles}.
É importante lembrar que a expressão do comportamento não está associado unicamente a movimento, mas a toda forma de expressão pelo animal. Por exemplo a bioluminescência de vaga-lumes (\textit{lampirídeos}), pirilampos (\textit{elaterídeos}) e bondinhos (\textit{fengodídeos}).

\subsection{Genética do comportamento - disciplina}
Enquanto disciplina, resulta da união das disciplinas de estudos do comportamento como  psicologia do comportamento e da etologia clássica com biologia evolutiva e genética, e também incorpora aspectos da neurociência.

A psicologia experimental é marcada pelas experiências do russo Igor Pavlov, o qual realizou experiências com cães e pode ver o condicionamento da salivação a partir de estímulos auditivos e visuais. Pavlov chamou esse comportamento condicionado dos cães de \textbf{reflexos condicionais} referidos como \textbf{condicionamento clássico} ou \textbf{aprendizagem associativa}.


Ao contrário dos estudos realizados em laboratórios, Konrad Lorenz e Karl von Frisch, e Nikolaas Tinbergen, começaram a aplicar abordagens experimentais a animais em seu habitat. Juntos, eles lançaram as bases para a etologia.% Essa abordagem tem a vantagem de observar o animal em seu próprio habitat. Suas abordagens experimentais pioneiras para descobrir princípios fundamentais que se aplicariam não apenas a uma única espécie foi tão relevante que lhes rendeu um Prêmio Nobel em 1973.
%Lorenz notou e apresentou também a existência de comportamentos padrões fixos, comportamentos instintivos, como por exemplo os rituais de acasalamento ou a construção de ninhos de pássaros. Esses comportamentos  estereotipados são acionados por um mecanismo de liberação inato, que induz uma sequência fixa de eventos comportamentais.
%Karl von Frisch mostrou esse comportamento a partir da dança da abelha que realiza uma sequência de manobras específicas como sinalização de que há alimento.
Por eles foram apresentados comportamentos instintivos/comportamentos padrões, como por exemplo rituais de acasalamento ou a construção de ninhos de pássaros ou a dança das abelhas.

No campo da genética, no século XIX, surgem as primeiras ideias evolucionistas com Lamarck em 1809 propondo que mudanças benéficas adquiridas durante a vida de um organismo poderia ser passada para sua progênie e que, ao longo de sucessivas gerações, esse processo altera as características do organismo teoria do ``uso/desuso''. Posteriormente Darwin apresenta uma teria mais completa para a evolução das espécies complementada pela seleção natural.
Gregor	Mendel descreve a lei da hereditariedade.Thomas Hunt Morgan demonstra o mapeamento de cromossomos.

Nas primeiras décadas do século XX são descobertos os neurotransmissores acetilcolina e adrenalina e passam a ser usados aparelhos eletrônicos para estudo da eletrofisiologia impulsionando os estudos em neurociência.

\subsection{Genética e hereditariedade}

%Parte do texto dessa seção e as imagens foram retiradas de \cite{dna}

O desenvolvimento de um organismo se dá basicamente pela influência de fatores inatos ao organismo e fatores ambientais no qual ocorre o desenvolvimento desse organismo. Richard Mulcaster usou o termo ``\textit{Nature through Nurture}'' para ilustrar a interação dessas componentes, ``\textit{nature}'' representa os fatores inatos e ``\textit{nurture}'' os fatores ambientais.

A componente biológica, isto é, a genética, é transferida dos progenitores à prole pelos cromossomos.
Eles são constituídos de DNA em forma de espiral contendo milhares genes.
Cromossomos homólogos possuem genes que determinam certa característica, organizados na mesma sequência. Os genes são  fragmentos sequenciais do DNA, responsáveis por codificar informações que irão determinar a produção de proteínas que atuarão na expressão das características de cada ser vivo. São a unidade fundamental da hereditariedade.

Os genes alelos, que ocupam o mesmo lugar (\textit{locus}) em cromossomos homólogos estão envolvidos na determinação de um mesmo caráter, cor dos olhos, por exemplo.

A constituição genética do indivíduo é representada por seu genótipo enquanto a expressão desses genes é denominada fenótipo.

Estudos utilizando gêmeos são importantes pois evidenciam a influência de cada fator sobre um caráter uma vez que o genoma é idêntico para ambos os indivíduos, alterando-se apenas fatores ambientais. Esse tipo de pesquisa evidenciou a grande influência da genética em distúrbios como esquizofrenia e autismo.

Todavia, para análise de outros aspectos relacionados a mutações e uso de organismos transgênicos, uma abordagem reducionistas (animais são menos complexos e tem caráteres associados a um simples genes) torna mais fácil o estudo. E partindo da teoria da evolução, encontrando um ancestral em comum pode-se estender as descobertas, dadas as devidas proporções, a organismos mais complexos. Outro fator importante para a utilização de modelos animais é o tamanho, o ciclo de vida e questões éticas.

Metodologias
\begin{itemize}
    \item Sintenia:	a	co-localização	de	genes	em cromossomos	de	espécies relacionadas;
    \item Homologia:	sequencias	de	DNA	ou proteína	que	são similares	por	 possuírem	um	origem	evolucionaria comum. Não significa igualdade, simplesmente origem comum;
    \begin{itemize}
            \item Ortólogos:	genes	com	sequências	 similares	entre espécies; e
            \item Parálogos: genes	com	sequências	 similares	na	mesma espécie.
    \end{itemize}
\end{itemize}
	

A genética dos organismos pode ser alterada por processos mutagênicos ou de transgênicos. O que difere ambos os processos é como a alteração ocorre. Mutagêneses é a ocorrência ou indução de uma mutação (gene já presente). Transgêneses ocorre com a transferência de genes externos para animais ou células.

Análise de mutações é ferramenta importante para o estudo da genética. Podemos abordá-la de duas formas distintas:
\begin{itemize}
    \item \textit{forward genetic screens}
    \item \textit{reverse genetic screens}
\end{itemize}
No primeiro cenário organismos mutantes são isolados e produzem certos fenótipos.
No segundo mutações são introduzidas em genes conhecidos para determinar quais fenótipos resultam como consequência.

Existem diversas técnicas para realizar essas alterações genéticas. Como a exposição do animal a irradiação ultravioleta ou raio-X, mutação por agentes químicos.
A mutagênese quimicamente induzida tem sido usada mais amplamente em muitos organismos modelo, pois gera mutações pontuais em genes específicos.

Estudos de características emocionais com camundongos devido a facilidade de observação do comportamento de exploração,	atividade,	ansiedade,	medo	e emoção/reação	por meio de	um teste de	campo	aberto	(\textit{Open	Field Apparatus}). Baixa	atividade	e	alta taxa	de	defecação	define	 reações	emotivas;	 medidas	geneticamente correlacionadas	(Animais	com	alta taxa	de	defecação	são	também	inativos).	

Camundongo	transgênico	são usados	para	mostrar	que	a	\textit{over}-expressão	de	um	gene	afeta	a	fisiologia,	o	
comportamento,	etc. Um	transgênico	pode ser	 gerado	entre	3-6	meses.

Mutação	dirigida	–	alteração	de	um	gene	de	interesse	por recombinação	homóloga. O	gene	pode	ser	eliminado (\textit{knock-out}	ou	mutante	nulo)	ou	alterado (\textit{knock-in}).		A	perda	ou	alteração	de	função	é	tipicamente	examinada	em	KO	ou KI. Pode	demorar	de	1-3	anos para	gerar	um	KO	ou	KI.

\subsection{Algumas doenças relacionadas à genética}


A Doença de Huntington (DH), uma doença neurodegenerativa hereditária e autossômica (não presente nos cromossomos não sexuais), dominante (basta estar presente em um alelo), causada pela repetição da sequência CAG. O número de repetições define o fenótipo. Mutação por expansão da repetição de CAG no gene huntingtina (cromossomo 4) e não simplesmente a repetição no genoma como um todo.
Indivíduos apresentam:
\begin{itemize}

    \item desordens motoras
    	\begin{itemize}
			\item  movimentos involuntários (movimentos bruscos ou contorcidos)
			\item  problemas musculares, como rigidez ou contratura muscular
			\item  movimentos oculares lentos ou anormais
			\item  dureza na marcha, postura e equilíbrio
			\item  dificuldade com a produção física de fala ou deglutição
    	\end{itemize}
    
    \item distúrbios não motores associados
        \begin{itemize}
			\item dificuldade em organizar, priorizar ou focar em tarefas
			\item falta de flexibilidade ou a tendência a ficar preso a um pensamento, comportamento ou ação (perseveração) 
			\item falta de controle de impulso que pode resultar em explosões, agindo sem pensar e promiscuidade sexual
			\item falta de consciência dos próprios comportamentos e habilidades
			\item lentidão no processamento de pensamentos ou ``encontrar'' palavras
			\item dificuldade em aprender novas informações
		\end{itemize}
		
\end{itemize}

A Doença de Parkinson é uma doença degenerativa do sistema nervoso central, crônica e progressiva. É causada por uma diminuição intensa da produção de dopamina, que é um neurotransmissor (substância química que ajuda na transmissão de mensagens entre as células nervosas). Fatores genéticos representam aproximadamente 15\%. Causada pelas mutações LRRK2,	PARK2,	PARK7,	PINK1	e/ou SNCA	(aka	PARK1). LRKK2	mutado	em	3-6\%	em	familiares	de	pacientes. Estão associados também os genes GBA,	SNCAIP	e	UCHL1.

A maioria dos casos de Parkinson são esporádicos (sem histórico familiar). Porém, descrito	pela primeira vez em uma	família	italiana:	mutação	no gene $\alpha-$sinucleína;
Indivíduos apresentam:
\begin{itemize}

    \item desordens motoras
    	\begin{itemize}
			\item  redução do estímulo	do córtex	motor	pelos gânglios basais
			\item  redução da formação e da ação da dopamina
			\item  rigidez	muscular,	tremor
    	\end{itemize}
    
    \item distúrbios não motores associados
        \begin{itemize}
			\item Súbitos	problemas	de	linguagem
            \item diminuição do olfato
            \item alterações intestinais
            \item alterações do sono
		\end{itemize}
    
\end{itemize}

\bookmarksetup{startatroot}% 
% ---
% ---
% Conclusão
% ---
\section*{Considerações finais}
\addcontentsline{toc}{section}{Considerações finais}

A disciplina de ``Genética do comportamento animal'' aborda de maneira multidisciplinar os fatores envolvidos no desenvolvimento e na expressão do comportamento. Mostra-se importante instrumento no aumento da compreensão de de outras áreas do conhecimento ao passo que introduz, devido ao seu caráter transdisciplinar,  novas ferramentas e descobertas para o tratamento de doenças. Cursar esse módulo foi importante para mim pois a cada aula me acrescentou uma bagagem das ciências biológicas que o curso de exatas não proporciona. Me fez repensar muito dos conceitos que tinha sobre genética e sobretudo sobre expressão e comportamento. Informações como a diferença no funcionamento dos órgãos dos sentidos, na grande quantidade de tipos de receptores que temos para a olfação se comparado com a gustação.

As técnicas apresentadas para localização dos genes ou \textit{locus} onde se encontram as informações para determinada expressão. Saber que nem todos os genes realmente expressão algum caráter foi uma informação muito nova, entender que nem toda a informação presente no DNA é lida que mesmo o código existindo nem sempre é expresso e que as vezes doenças podem estar lá e não se manifestarem ou alterações no ambiente podem realmente fazer um gene ``inativo'' gerar alguma desordem, faz repensar os dispositivos e aparelhos que podem e devem ser desenvolvidos na área biomédica e, sobretudo, a responsabilidade de buscar entender a interação desses aparelhos com os pacientes e os profissionais que os utilizarão.

Embora tenha ido cursar o módulo na busca por informações na área do movimento e reabilitação especificamente. Pude aprender muito sobre diversas formas de utilizar o conhecimento e as ferramentas do estudo da genética e do comportamento no auxílio e tratamento de desordens associadas a perda de função de pacientes.


Algumas informações contidas nesse trabalho foram adaptadas de \cite{einstein}, \cite{dna} e \cite{macdonald1993novel}.

% ----------------------------------------------------------
% ELEMENTOS PÓS-TEXTUAIS
% ----------------------------------------------------------
\postextual

% ----------------------------------------------------------
% Referências bibliográficas
% ----------------------------------------------------------
\bibliography{abntex2-modelo-references}

\end{document}
